\chapter{RESEARCH METHODOLOGY}
\label{ch:method}

\section{Introduction}
This section provides an in-depth description of the design and development characteristics of Alunan, including the tools, processes, and techniques used. The chapter additionally outlines the research technique utilized in the development of the Alunan application and its subsequent implementation inside the project. This project follows the five stages of the Mobile Application Development Lifecycle (MADLC).

\section{Overview of Mobile Application Development Lifecycle (MADLC)}
insert Methodology diagram

\section{Methodology Development and Related Activities}

\subsection{Identification Phase}
\begin{table}[htb]
\caption{Overview of Identification Phase} 
\label{tab:mytable}
\centering
\begin{tabular}{|p{2.2cm}|p{2.6cm}|p{2.6cm}|p{2.6cm}|p{2.6cm}|}
\hline
\multicolumn{1}{|c|}{\textbf{Phase}} & 
\multicolumn{1}{c|}{\textbf{Objectives}} & 
\multicolumn{1}{c|}{\textbf{Activities}} & 
\multicolumn{1}{c|}{\textbf{Tools \& Techniques}} & 
\multicolumn{1}{c|}{\textbf{Deliverables}} \\
\hline 
\multirow{3}{*}{%
	\begin{tabular}[c]{@{}p{2.2cm}@{}}
	\vspace{4.4cm} \raggedright Identification \\[6pt]
	\end{tabular}
} &
\multirow{3}{*}{%
	\begin{tabular}[c]{@{}p{2.6cm}@{}}
	\vspace{1.8cm} \raggedright To identify system requirements for Alunan as a mobile application for local independent musicians' online community and music discovery \\[6pt]
	\end{tabular}
} &
\multirow{1}{*}{%
	\begin{tabular}[c]{@{}p{2.6cm}@{}}
	\vspace{0.5cm} \raggedright Collecting, gathering information and identifying the problem, objective, scope, and significance \\[6pt]
	\end{tabular}
} &
\raggedright Techniques: Literature Review \newline \newline Tools: Online Database UiTM, ResearchGate, IEEE Xplore, ScienceDirect, Scopus, \& Google Scholar &
\multirow{1}{*}{%
	\begin{tabular}[c]{@{}p{2.6cm}@{}}
	\vspace{0.5cm} \raggedright System requirements, Problem Statement, Objectives, Scope \& Significance \\[6pt]
	\end{tabular}
} \\ \cline{3-5}
& &
\multirow{1}{*}{%
	\begin{tabular}[c]{@{}p{2.6cm}@{}}
	\vspace{-0.1cm} \raggedright Define stakeholder characteristics \\[6pt]
	\end{tabular}
} &
\raggedright Technique: User Persona \newline \newline Tools: Canva & 
\multirow{1}{*}{%
	\begin{tabular}[c]{@{}p{2.6cm}@{}}
	\vspace{0.35cm} \raggedright User Persona \\[6pt]
	\end{tabular}
} \\ \cline{3-5}
& &
\multirow{1}{*}{%
	\begin{tabular}[c]{@{}p{2.6cm}@{}}
	\vspace{0.55cm} \raggedright Plan the timeline of the project \\[6pt]
	\end{tabular}
} &
\raggedright Techniques: Gantt Chart \newline \newline Tools: Lucidchart \& Canva & 
\multirow{1}{*}{%
	\begin{tabular}[c]{@{}p{2.6cm}@{}}
	\vspace{0.55cm} \raggedright Project Timeline (Gantt Chart) \\[6pt]
	\end{tabular}
} \\ \hline
\end{tabular}
\end{table}	
\pagebreak

\subsection{Design Phase}
\begin{spacing}{1.0}
\begin{longtable}{|p{2.2cm}|p{2.6cm}|p{2.6cm}|p{2.6cm}|p{2.6cm}|}
\caption{Overview of Design Phase} 
\label{tab:mytable}\\
\hline
\multicolumn{1}{|c|}{\textbf{Phase}} & 
\multicolumn{1}{c|}{\textbf{Objectives}} & 
\multicolumn{1}{c|}{\textbf{Activities}} & 
\multicolumn{1}{c|}{\textbf{Tools \& Techniques}} & 
\multicolumn{1}{c|}{\textbf{Deliverables}} \\
\hline 
\endfirsthead % header for the first page
\multicolumn{5}{c}{{\tablename\ \thetable{} -- Continued from previous page}} \\
\hline
\multicolumn{1}{|c|}{\textbf{Phase}} & 
\multicolumn{1}{c|}{\textbf{Objectives}} & 
\multicolumn{1}{c|}{\textbf{Activities}} & 
\multicolumn{1}{c|}{\textbf{Tools \& Techniques}} & 
\multicolumn{1}{c|}{\textbf{Deliverables}} \\
\hline 
\endhead % header for subsequent pages
\multirow{7}{*}{%
	\begin{tabular}[c]{@{}p{2.2cm}@{}}
	\vspace{4.4cm} \raggedright Design \\[6pt]
	\end{tabular}
} &
\multirow{7}{*}{%
	\begin{tabular}[c]{@{}p{2.6cm}@{}}
	\vspace{1.8cm} \raggedright To design Alunan as a mobile application for local independent musicians’ online community and music discovery \\[6pt]
	\end{tabular}
} &
\multirow{1}{*}{%
	\begin{tabular}[c]{@{}p{2.6cm}@{}}
	\vspace{0.5cm} \raggedright Review existing apps and identify the hardware and software requirements \\[6pt]
	\end{tabular}
} &
\raggedright Techniques: Literature and research summary \newline \newline Tools: Google Play Store, Apple App Store, Microsoft Word, Laptop, Mobile Phone \& Google Chrome &
\multirow{1}{*}{%
	\begin{tabular}[c]{@{}p{2.6cm}@{}}
	\vspace{0.5cm} \raggedright Summary of design features and functions of existing applications \newline \newline Hardware and Software Requirements\\[6pt]
	\end{tabular}
} \\ \cline{3-5}
& &
\multirow{1}{*}{%
	\begin{tabular}[c]{@{}p{2.6cm}@{}}
	\vspace{-0.1cm} \raggedright Outline system features \\[6pt]
	\end{tabular}
} &
\raggedright Technique: Requirement Analysis \newline \newline Tools: Google Chrome, Microsoft Word \& Canva & 
\multirow{1}{*}{%
	\begin{tabular}[c]{@{}p{2.6cm}@{}}
	\vspace{0.35cm} \raggedright Functional Requirements \& Non-Functional Requirements \\[6pt]
	\end{tabular}
} \\ \cline{3-5}
& &
\multirow{1}{*}{%
	\begin{tabular}[c]{@{}p{2.6cm}@{}}
	\vspace{0.55cm} \raggedright Designing User Interface \\[6pt]
	\end{tabular}
} &
\raggedright Techniques: Storyboard \& Wireframe \newline \newline Tools: Canva \& Figma & 
\multirow{1}{*}{%
	\begin{tabular}[c]{@{}p{2.6cm}@{}}
	\vspace{0.55cm} \raggedright Storyboard \& Wireframe \\[6pt]
	\end{tabular}
} \\ \cline{3-5}
& &
\multirow{4}{*}{%
	\begin{tabular}[c]{@{}p{2.6cm}@{}}
	\vspace{0.55cm} \raggedright Designing interaction flow for the app \\[6pt]
	\end{tabular}
} &
\raggedright Techniques: Flowchart \newline \newline Tools: Lucidchart \& Draw.io & 
\multirow{1}{*}{%
	\begin{tabular}[c]{@{}p{2.6cm}@{}}
	\vspace{0.55cm} \raggedright Flowcharts of the app \\[6pt]
	\end{tabular}
} \\ \cline{4-5}
& & &
\raggedright Techniques: Use Case Diagram (UCD) \newline \newline Tools: Lucidchart \& Draw.io & 
\multirow{1}{*}{%
	\begin{tabular}[c]{@{}p{2.6cm}@{}}
	\vspace{0.55cm} \raggedright Use Case Diagram (UCD) \\[6pt]
	\end{tabular}
} \\ \cline{4-5}
& & &
\raggedright Techniques: Entity Relationship Diagram (ERD) \newline \newline Tools: Lucidchart \& Draw.io & 
\multirow{1}{*}{%
	\begin{tabular}[c]{@{}p{2.6cm}@{}}
	\vspace{0.55cm} \raggedright Entity Relationship Diagram (ERD) \\[6pt]
	\end{tabular}
} \\ \cline{4-5}
& & &
\raggedright Techniques: Sequence Diagram \newline \newline Tools: Lucidchart \& Draw.io & 
\multirow{1}{*}{%
	\begin{tabular}[c]{@{}p{2.6cm}@{}}
	\vspace{0.55cm} \raggedright Sequence Diagram \\[6pt]
	\end{tabular}
} \\ \hline
\end{longtable}
\end{spacing}

\subsection{Development Phase}
insert table

\subsection{Prototyping Phase}
insert table

\subsection{Testing Phase}
insert table

\section{Study Area}

\section{Sampling}


\begin{table}[ht]
    \caption{My Sample}
    \begin{tabular}{cc}
        \toprule %header
        \textbf{Millimeters} & \textbf{Centimeters}\\
        \textbf{mm}          &   \textbf{cm}\\
        \midrule
        1           &   0.1\\ \hline
        10          &   1\\ \hline
        100         &   10\\ \hline
        1000        &   100\\ \hline
        10000       &   1000\\
        \bottomrule
    \end{tabular}
    \par\raggedright Note: This table is useful for $\ldots$.
    \label{tab:my_label}
\end{table}

\begin{table}[ht]
    \caption{The Second Sample}
    \begin{tabular}{>{\centering\arraybackslash}p{.47\textwidth} >{\centering\arraybackslash}p{.47\textwidth}}
        \toprule %header
        \textbf{Millimeters} & \textbf{Centimeters}\\
        \textbf{mm}          &   \textbf{cm}\\
        \midrule
        1           &   0.1\\ \hline
        10          &   1\\ \hline
        100         &   10\\ \hline
        1000        &   100\\ \hline
        10000       &   1000\\
        \bottomrule
    \end{tabular}
    \par\raggedright Note: This table is useful for $\ldots$.
    \label{tab:my_second_label}
\end{table}

\begin{figure}[ht]
    \centering
    \fbox{%
        \includegraphics{mainmatter/images/logouitm.png}
    }
    \caption{A New Figure Again!}
    \label{fig:newfig}
\end{figure}

\lipsum[2]

\begin{figure}[ht]
    \centering
    \begin{tabular}{|c|c|}
    \hline
     \begin{subfigure}[b]{0.44\textwidth}
         \centering
         \includegraphics[width=.8\linewidth]{mainmatter/images/logouitm.png}
         \caption{$y=x$}
         \label{fig:y_equals_x}
     \end{subfigure} &
     \begin{subfigure}[b]{0.44\textwidth}
         \centering
         \includegraphics[width=.8\linewidth]{mainmatter/images/logouitm.png}
         \caption{$x=y$}
         \label{fig:x_equals_y}
     \end{subfigure} \\
     \hline
    \end{tabular}
    \caption{The Two Figures}
    \label{fig:the2fig}
\end{figure}

\lipsum[1]

\begin{figure}[ht]
    \centering
	\begin{tabular}{|c|c|}
		\hline
		\begin{subfigure}[b]{0.44\textwidth}
			\centering
			\includegraphics[width=.8\linewidth]{mainmatter/images/logouitm.png}
			\caption{$x=y$}
         	\label{fig:x_equals_y4}
		\end{subfigure} & 
				\begin{subfigure}[b]{0.44\textwidth}
			\centering
			\includegraphics[width=.8\linewidth]{mainmatter/images/logouitm.png}
			\caption{$x=y$}
         	\label{fig:xy_equals_y3}
		\end{subfigure}	\\
		\hline
		\begin{subfigure}[b]{0.44\textwidth}
			\centering
			\includegraphics[width=.8\linewidth]{mainmatter/images/logouitm.png}
			\caption{$x=y$}
         	\label{fig:yy1}
		\end{subfigure} & 
				\begin{subfigure}[b]{0.44\textwidth}
			\centering
			\includegraphics[width=.8\linewidth]{mainmatter/images/logouitm.png}
			\caption{$x=y$}
         	\label{fig:xy2}
		\end{subfigure}	\\
		\hline		
	\end{tabular}
    \caption{The Four Figures}
    \label{fig:the4fig}
\end{figure}


\begin{landscape}

\begin{figure}[ht]
    \centering
	\begin{tabular}{|c|c|c|}
		\hline
		\begin{subfigure}[b]{0.44\textwidth}
			\centering
			\includegraphics[width=.8\linewidth]{example-image-a.jpg}
			\caption{$x=y$}
         	\label{fig:ly1}
		\end{subfigure} & 
		\begin{subfigure}[b]{0.44\textwidth}
			\centering
			\includegraphics[width=.8\linewidth]{example-image-b.jpg}
			\caption{$x=y$}
         	\label{fig:ly2}
		\end{subfigure} & 
		\begin{subfigure}[b]{0.44\textwidth}
			\centering
			\includegraphics[width=.8\linewidth]{example-image-c.jpg}
			\caption{$x=y$}
         	\label{fig:ly3}
		\end{subfigure}\\
		\hline
		\begin{subfigure}[b]{0.44\textwidth}
			\centering
			\includegraphics[width=.8\linewidth]{mainmatter/images/logouitm.png}
			\caption{$x=y$}
         	\label{fig:ly4}
		\end{subfigure} & 
		\begin{subfigure}[b]{0.44\textwidth}
			\centering
			\includegraphics[width=.8\linewidth]{mainmatter/images/logouitm.png}
			\caption{$x=y$}
         	\label{fig:ly5}
		\end{subfigure} & 
		\begin{subfigure}[b]{0.44\textwidth}
			\centering
			\includegraphics[width=.8\linewidth]{mainmatter/images/logouitm.png}
			\caption{$x=y$}
         	\label{fig:ly6}
		\end{subfigure}\\
		\hline
		\begin{subfigure}[b]{0.44\textwidth}
			\centering
			\includegraphics[width=.8\linewidth]{example-image-a.jpg}
			\caption{$x=y$}
         	\label{fig:ly7}
		\end{subfigure} & 
		\begin{subfigure}[b]{0.44\textwidth}
			\centering
			\includegraphics[width=.8\linewidth]{example-image-b.jpg}
			\caption{$x=y$}
         	\label{fig:ly8}
		\end{subfigure} & 
		\begin{subfigure}[b]{0.44\textwidth}
			\centering
			\includegraphics[width=.8\linewidth]{example-image-c.jpg}
			\caption{$x=y$}
         	\label{fig:ly9}
		\end{subfigure}\\
		\hline
	\end{tabular}	      
    \caption{The Landscape Figures}
    \label{fig:my_label_landscape}
\end{figure}

\end{landscape}

\lipsum[1]