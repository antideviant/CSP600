\chapter{RESEARCH METHODOLOGY}
\label{ch:method}

\section{Introduction}
This section provides an in-depth description of the design and development characteristics of Alunan, including the tools, processes, and techniques used. The chapter additionally outlines the research technique utilized in the development of the Alunan application and its subsequent implementation inside the project. This project follows the five stages of the Mobile Application Development Lifecycle (MADLC).

\section{Overview of Mobile Application Development Lifecycle (MADLC)}
The methodology, as defined by \textcite{igwenagu16}, relates to a systematic and detailed analysis of the steps used in a certain area of research. Within the realm of mobile application development, the technique functions as a structured framework for efficiently implementing an application. The methodology offers a systematic plan, outlining the sequential processes, tasks, techniques, and requirements entailed in the creation of an application. This chapter will discuss the research approach utilized in this project and its implementation. The selected methodology for this project is the Mobile Application Development Lifecycle (MADLC).

\begin{figure}[h]
    \centering
    \includegraphics[width=0.8\linewidth]{mainmatter/images/madlc1.png}
    \caption{MADLC Phases Included in the Project}
    \caption*{Source: \textcite{moharekar21}}
    \caption*{\textit{Appropriated by: \textcite{shanmugam19}}}
    \label{fig:myfig29}
\end{figure}

The Mobile Application Development Life Cycle (MADLC) is a systematic approach introduced by \textcite{vithani14} to address various issues presented by mobile applications. These apps frequently include complex functionality that differs greatly from conventional desktop applications. The introduction of MADLC aimed to establish a structured framework for the development of mobile apps, recognizing the necessity for a customized approach \parencite{vithani14}. \\

\textcite{kaur15} addressed the significance of a dedicated development strategy for mobile applications, acknowledging their unique characteristics. \textcite{kaur15} also emphasize the necessity of adopting a unique methodology to address the unique characteristics and needs of mobile app development. MADLC addresses this requirement by providing a systematic and sequential approach that corresponds to the distinct difficulties and characteristics of mobile app development, including diverse screen sizes, device functionalities, and user expectations. \\

The Mobile Application Development Life Cycle (MADLC) consists of multiple stages that provide a structured framework for the development of mobile apps, as outlined by \textcite{wambua23} and explained by \textcite{wen21}. The following stages are crucial for developing mobile applications that achieve success:

\begin{enumerate}[1.]
    \item \textit{Identification:} The identification step involves identifying the problem statement and establishing the goal and objectives for creating the mobile application. System requirements are collected and the target audience is determined. Market research is performed to evaluate the potential viability of the application, and specific project scope and objectives are established.
    \item \textit{Design:} After the identification phase, the design phase begins. Developers are responsible for developing the user interface (UI) and user experience (UX) elements of the application. During this phase, flowcharts, Use Case Diagrams (UCD), Entity Relationship Diagrams (ERD), and Sequence Diagrams are generated. Wireframes, mockups, and prototypes are created to visually represent the structure and user experience. The selection of the suitable technology stack, databases, and architecture for the mobile application has been made.
    \item \textit{Development:} During the development phase, developers begin coding the mobile application, building its functionality according to the previously gathered design and specifications. This phase involves incorporating functionalities and ensuring consistency with the design and user experience principles. If necessary, backend systems, APIs, and server components are developed.
    \item \textit{Prototyping:} During this stage, the process of combining different systems and coding the front end and back end happens. Developers create a high-fidelity prototype of the mobile application, which includes important features and functions. The prototype plays a vital role in the development process by enabling the examination and verification of concepts and features. Early user feedback is sought and utilized to identify and address any design or functionality concerns before moving forward.
    \item \textit{Testing:} Quality assurance and testing are essential elements of MADLC. During this phase, a range of testing methods are employed, including functional testing, usability testing, performance testing, security testing, and compatibility testing across many devices and platforms. Testers detect and document defects and problems, which the developer then resolves to guarantee that the mobile application fulfills its functional specifications and provides a smooth user interface.
\end{enumerate}

These phases offer a systematic approach to the creation of mobile applications, guaranteeing that the outcome is in line with objectives, displays a strong design, undergoes comprehensive testing, and fulfills the requirements of its intended users. Nevertheless, it is critical to acknowledge that the MADLC has seven essential stages: identification, design, development, prototyping, testing, deployment, and maintenance. Within the scope of this project, the primary focus is on the early stages of the MADLC, specifically covering the identification to testing phases.
\pagebreak

\section{Methodology Development and Related Activities}
\subsection{Identification Phase}
\begin{table}[htb]
\caption{Overview of Identification Phase} 
\label{tab:mytable}
\centering
\begin{tabular}{|p{2.2cm}|p{2.6cm}|p{2.6cm}|p{2.6cm}|p{2.6cm}|}
\hline
\multicolumn{1}{|c|}{\textbf{Phase}} & 
\multicolumn{1}{c|}{\textbf{Objectives}} & 
\multicolumn{1}{c|}{\textbf{Activities}} & 
\multicolumn{1}{c|}{\textbf{Tools \& Techniques}} & 
\multicolumn{1}{c|}{\textbf{Deliverables}} \\
\hline 
\multirow{3}{*}{%
	\begin{tabular}[c]{@{}p{2.2cm}@{}}
	\vspace{4.4cm} \raggedright Identification \\[6pt]
	\end{tabular}
} &
\multirow{3}{*}{%
	\begin{tabular}[c]{@{}p{2.6cm}@{}}
	\vspace{1.8cm} \raggedright To identify system requirements for Alunan as a mobile application for local independent musicians' online community and music discovery \\[6pt]
	\end{tabular}
} &
\multirow{1}{*}{%
	\begin{tabular}[c]{@{}p{2.6cm}@{}}
	\vspace{0.5cm} \raggedright Collecting, gathering information and identifying the problem, objective, scope, and significance \\[6pt]
	\end{tabular}
} &
\raggedright Techniques: Literature Review \newline \newline Tools: Online Database UiTM, ResearchGate, IEEE Xplore, ScienceDirect, Scopus, \& Google Scholar &
\multirow{1}{*}{%
	\begin{tabular}[c]{@{}p{2.6cm}@{}}
	\vspace{0.5cm} \raggedright System requirements, Problem Statement, Objectives, Scope \& Significance \\[6pt]
	\end{tabular}
} \\ \cline{3-5}
& &
\multirow{1}{*}{%
	\begin{tabular}[c]{@{}p{2.6cm}@{}}
	\vspace{-0.1cm} \raggedright Define stakeholder characteristics \\[6pt]
	\end{tabular}
} &
\raggedright Technique: User Persona \newline \newline Tools: Canva & 
\multirow{1}{*}{%
	\begin{tabular}[c]{@{}p{2.6cm}@{}}
	\vspace{0.35cm} \raggedright User Persona \\[6pt]
	\end{tabular}
} \\ \cline{3-5}
& &
\multirow{1}{*}{%
	\begin{tabular}[c]{@{}p{2.6cm}@{}}
	\vspace{0.55cm} \raggedright Plan the timeline of the project \\[6pt]
	\end{tabular}
} &
\raggedright Techniques: Gantt Chart \newline \newline Tools: Lucidchart \& Canva & 
\multirow{1}{*}{%
	\begin{tabular}[c]{@{}p{2.6cm}@{}}
	\vspace{0.55cm} \raggedright Project Timeline (Gantt Chart) \\[6pt]
	\end{tabular}
} \\ \hline
\end{tabular}
\end{table}	
\pagebreak

\subsection{Design Phase}
\begin{spacing}{1.0}
\begin{longtable}{|p{2.2cm}|p{2.6cm}|p{2.6cm}|p{2.6cm}|p{2.6cm}|}
\caption{Overview of Design Phase} 
\label{tab:mytable}\\
\hline
\multicolumn{1}{|c|}{\textbf{Phase}} & 
\multicolumn{1}{c|}{\textbf{Objectives}} & 
\multicolumn{1}{c|}{\textbf{Activities}} & 
\multicolumn{1}{c|}{\textbf{Tools \& Techniques}} & 
\multicolumn{1}{c|}{\textbf{Deliverables}} \\
\hline 
\endfirsthead % header for the first page
\multicolumn{5}{c}{{\tablename\ \thetable{} -- Continued from previous page}} \\
\hline
\multicolumn{1}{|c|}{\textbf{Phase}} & 
\multicolumn{1}{c|}{\textbf{Objectives}} & 
\multicolumn{1}{c|}{\textbf{Activities}} & 
\multicolumn{1}{c|}{\textbf{Tools \& Techniques}} & 
\multicolumn{1}{c|}{\textbf{Deliverables}} \\
\hline 
\endhead % header for subsequent pages
\multirow{7}{*}{%
	\begin{tabular}[c]{@{}p{2.2cm}@{}}
	\vspace{4.4cm} \raggedright Design \\[6pt]
	\end{tabular}
} &
\multirow{7}{*}{%
	\begin{tabular}[c]{@{}p{2.6cm}@{}}
	\vspace{1.8cm} \raggedright To design Alunan as a mobile application for local independent musicians’ online community and music discovery \\[6pt]
	\end{tabular}
} &
\multirow{1}{*}{%
	\begin{tabular}[c]{@{}p{2.6cm}@{}}
	\vspace{0.5cm} \raggedright Review existing apps and identify the hardware and software requirements \\[6pt]
	\end{tabular}
} &
\raggedright Techniques: Literature and research summary \newline \newline Tools: Google Play Store, Apple App Store, Microsoft Word, Laptop, Mobile Phone \& Google Chrome &
\multirow{1}{*}{%
	\begin{tabular}[c]{@{}p{2.6cm}@{}}
	\vspace{0.5cm} \raggedright Summary of design features and functions of existing applications \newline \newline Hardware and Software Requirements\\[6pt]
	\end{tabular}
} \\ \cline{3-5}
& &
\multirow{1}{*}{%
	\begin{tabular}[c]{@{}p{2.6cm}@{}}
	\vspace{-0.1cm} \raggedright Outline system features \\[6pt]
	\end{tabular}
} &
\raggedright Technique: Requirement Analysis \newline \newline Tools: Google Chrome, Microsoft Word \& Canva & 
\multirow{1}{*}{%
	\begin{tabular}[c]{@{}p{2.6cm}@{}}
	\vspace{0.35cm} \raggedright Functional Requirements \& Non-Functional Requirements \\[6pt]
	\end{tabular}
} \\ \cline{3-5}
& &
\multirow{1}{*}{%
	\begin{tabular}[c]{@{}p{2.6cm}@{}}
	\vspace{0.55cm} \raggedright Designing User Interface \\[6pt]
	\end{tabular}
} &
\raggedright Techniques: Storyboard \& Wireframe \newline \newline Tools: Canva \& Figma & 
\multirow{1}{*}{%
	\begin{tabular}[c]{@{}p{2.6cm}@{}}
	\vspace{0.55cm} \raggedright Storyboard \& Wireframe \\[6pt]
	\end{tabular}
} \\ \cline{3-5}
& &
\multirow{4}{*}{%
	\begin{tabular}[c]{@{}p{2.6cm}@{}}
	\vspace{0.55cm} \raggedright Designing interaction flow for the app \\[6pt]
	\end{tabular}
} &
\raggedright Techniques: Flowchart \newline \newline Tools: Lucidchart \& Draw.io & 
\multirow{1}{*}{%
	\begin{tabular}[c]{@{}p{2.6cm}@{}}
	\vspace{0.55cm} \raggedright Flowcharts of the app \\[6pt]
	\end{tabular}
} \\ \cline{4-5}
& & &
\raggedright Techniques: Use Case Diagram (UCD) \newline \newline Tools: Lucidchart \& Draw.io & 
\multirow{1}{*}{%
	\begin{tabular}[c]{@{}p{2.6cm}@{}}
	\vspace{0.55cm} \raggedright Use Case Diagram (UCD) \\[6pt]
	\end{tabular}
} \\ \cline{4-5}
& & &
\raggedright Techniques: Entity Relationship Diagram (ERD) \newline \newline Tools: Lucidchart \& Draw.io & 
\multirow{1}{*}{%
	\begin{tabular}[c]{@{}p{2.6cm}@{}}
	\vspace{0.55cm} \raggedright Entity Relationship Diagram (ERD) \\[6pt]
	\end{tabular}
} \\ \cline{4-5}
& & &
\raggedright Techniques: Sequence Diagram \newline \newline Tools: Lucidchart \& Draw.io & 
\multirow{1}{*}{%
	\begin{tabular}[c]{@{}p{2.6cm}@{}}
	\vspace{0.55cm} \raggedright Sequence Diagram \\[6pt]
	\end{tabular}
} \\ \hline
\end{longtable}
\end{spacing}
\pagebreak

\subsection{Development Phase}
\begin{table}[htb]
\caption{Overview of Development Phase} 
\label{tab:mytable}
\centering
\begin{tabular}{|p{2.2cm}|p{2.6cm}|p{2.6cm}|p{2.6cm}|p{2.6cm}|}
\hline
\multicolumn{1}{|c|}{\textbf{Phase}} & 
\multicolumn{1}{c|}{\textbf{Objectives}} & 
\multicolumn{1}{c|}{\textbf{Activities}} & 
\multicolumn{1}{c|}{\textbf{Tools \& Techniques}} & 
\multicolumn{1}{c|}{\textbf{Deliverables}} \\
\hline 
\multirow{2}{*}{%
	\begin{tabular}[c]{@{}p{2.2cm}@{}}
	\vspace{2.3cm} \raggedright Development \\[6pt]
	\end{tabular}
} &
\multirow{2}{*}{%
	\begin{tabular}[c]{@{}p{2.6cm}@{}}
	\vspace{0.4cm} \raggedright To develop the Alunan as a mobile application for local independent musicians’ online community and music discovery \\[6pt]
	\end{tabular}
} &
\multirow{1}{*}{%
	\begin{tabular}[c]{@{}p{2.6cm}@{}}
	\vspace{0.45cm} \raggedright Develop user interface and functionalities \\[6pt]
	\end{tabular}
} &
\raggedright Techniques: User Interface Development \newline \newline Tools: Android Studio \& Figma &
\multirow{1}{*}{%
	\begin{tabular}[c]{@{}p{2.6cm}@{}}
	\vspace{-0.07cm} \raggedright User Interface for Alunan mobile application developed \\[6pt]
	\end{tabular}
} \\ \cline{3-5}
& &
\multirow{1}{*}{%
	\begin{tabular}[c]{@{}p{2.6cm}@{}}
	\vspace{0.35cm} \raggedright Establish database for the mobile application \\[6pt]
	\end{tabular}
} &
\raggedright Technique: Database Development \newline \newline Tools: Android Studio \& Firebase & 
\multirow{1}{*}{%
	\begin{tabular}[c]{@{}p{2.6cm}@{}}
	\vspace{0.38cm} \raggedright Database for Alunan mobile application created \\[6pt]
	\end{tabular}
} \\ \hline
\end{tabular}
\end{table}
\pagebreak

\subsection{Prototyping Phase}
\begin{table}[htb]
\caption{Overview of Prototyping Phase} 
\label{tab:mytable}
\centering
\begin{tabular}{|p{2.2cm}|p{2.6cm}|p{2.6cm}|p{2.6cm}|p{2.6cm}|}
\hline
\multicolumn{1}{|c|}{\textbf{Phase}} & 
\multicolumn{1}{c|}{\textbf{Objectives}} & 
\multicolumn{1}{c|}{\textbf{Activities}} & 
\multicolumn{1}{c|}{\textbf{Tools \& Techniques}} & 
\multicolumn{1}{c|}{\textbf{Deliverables}} \\
\hline 
\multirow{2}{*}{%
	\begin{tabular}[c]{@{}p{2.2cm}@{}}
	\vspace{3.4cm} \raggedright Prototyping \\[6pt]
	\end{tabular}
} &
\multirow{2}{*}{%
	\begin{tabular}[c]{@{}p{2.6cm}@{}}
	\vspace{1.4cm} \raggedright To develop the Alunan as a mobile application for local independent musicians’ online community and music discovery \\[6pt]
	\end{tabular}
} &
\multirow{1}{*}{%
	\begin{tabular}[c]{@{}p{2.6cm}@{}}
	\vspace{1.05cm} \raggedright System Integration and Frontend-Backend Coding \\[6pt]
	\end{tabular}
} &
\raggedright Techniques: Frontend and Backend coding and System Integration Coding \newline \newline Tools: Android Studio \& Firebase &
\multirow{1}{*}{%
	\begin{tabular}[c]{@{}p{2.6cm}@{}}
	\vspace{0.75cm} \raggedright Coding algorithms that integrate the frontend and backend elements \\[6pt]
	\end{tabular}
} \\ \cline{3-5}
& &
\multirow{1}{*}{%
	\begin{tabular}[c]{@{}p{2.6cm}@{}}
	\vspace{0.2cm} \raggedright Construct a high-fidelity Alunan app prototype with key features and interactions \\[6pt]
	\end{tabular}
} &
\raggedright Technique: High-Fidelity Prototyping \newline \newline Tools: \newline Android Studio, \newline Firebase \& \newline Tablet & 
\multirow{1}{*}{%
	\begin{tabular}[c]{@{}p{2.6cm}@{}}
	\vspace{-0.1cm} \raggedright A high-fidelity prototype of the Alunan mobile application that includes key features and interactions \\[6pt]
	\end{tabular}
} \\ \hline
\end{tabular}
\end{table}
\pagebreak

\subsection{Testing Phase}
\begin{table}[htb]
\caption{Overview of Testing Phase} 
\label{tab:mytable}
\centering
\begin{tabular}{|p{2.2cm}|p{2.6cm}|p{2.6cm}|p{2.6cm}|p{2.6cm}|}
\hline
\multicolumn{1}{|c|}{\textbf{Phase}} & 
\multicolumn{1}{c|}{\textbf{Objectives}} & 
\multicolumn{1}{c|}{\textbf{Activities}} & 
\multicolumn{1}{c|}{\textbf{Tools \& Techniques}} & 
\multicolumn{1}{c|}{\textbf{Deliverables}} \\
\hline 
\multirow{2}{*}{%
	\begin{tabular}[c]{@{}p{2.2cm}@{}}
	\vspace{3.4cm} \raggedright Testing \\[6pt]
	\end{tabular}
} &
\multirow{2}{*}{%
	\begin{tabular}[c]{@{}p{2.6cm}@{}}
	\vspace{1.4cm} \raggedright To develop the Alunan as a mobile application for local independent musicians’ online community and music discovery \\[6pt]
	\end{tabular}
} &
\multirow{1}{*}{%
	\begin{tabular}[c]{@{}p{2.6cm}@{}}
	\vspace{0.65cm} \raggedright Perform application testing \\[6pt]
	\end{tabular}
} &
\raggedright Techniques: Android Emulator \newline \newline Tools: Android Studio, Firebase \& Tablet &
\multirow{1}{*}{%
	\begin{tabular}[c]{@{}p{2.6cm}@{}}
	\vspace{0.78cm} \raggedright Result of Testing \\[6pt]
	\end{tabular}
} \\ \cline{3-5}
& &
\multirow{1}{*}{%
	\begin{tabular}[c]{@{}p{2.6cm}@{}}
	\vspace{1cm} \raggedright Perform usability and user acceptance testing \\[6pt]
	\end{tabular}
} &
\raggedright Technique: System Usability Scale (SUS) Questionnaire \newline \newline Tools: \newline Google Form \& \newline Google Spreadsheet & 
\multirow{1}{*}{%
	\begin{tabular}[c]{@{}p{2.6cm}@{}}
	\vspace{1.45cm} \raggedright SUS Score and User Feedback \\[6pt]
	\end{tabular}
} \\ \hline
\end{tabular}
\end{table}
\pagebreak

\section{Study Area}

\section{Sampling}


\begin{table}[ht]
    \caption{My Sample}
    \begin{tabular}{cc}
        \toprule %header
        \textbf{Millimeters} & \textbf{Centimeters}\\
        \textbf{mm}          &   \textbf{cm}\\
        \midrule
        1           &   0.1\\ \hline
        10          &   1\\ \hline
        100         &   10\\ \hline
        1000        &   100\\ \hline
        10000       &   1000\\
        \bottomrule
    \end{tabular}
    \par\raggedright Note: This table is useful for $\ldots$.
    \label{tab:my_label}
\end{table}

\begin{table}[ht]
    \caption{The Second Sample}
    \begin{tabular}{>{\centering\arraybackslash}p{.47\textwidth} >{\centering\arraybackslash}p{.47\textwidth}}
        \toprule %header
        \textbf{Millimeters} & \textbf{Centimeters}\\
        \textbf{mm}          &   \textbf{cm}\\
        \midrule
        1           &   0.1\\ \hline
        10          &   1\\ \hline
        100         &   10\\ \hline
        1000        &   100\\ \hline
        10000       &   1000\\
        \bottomrule
    \end{tabular}
    \par\raggedright Note: This table is useful for $\ldots$.
    \label{tab:my_second_label}
\end{table}

\begin{figure}[ht]
    \centering
    \fbox{%
        \includegraphics{mainmatter/images/logouitm.png}
    }
    \caption{A New Figure Again!}
    \label{fig:newfig}
\end{figure}

\lipsum[2]

\begin{figure}[ht]
    \centering
    \begin{tabular}{|c|c|}
    \hline
     \begin{subfigure}[b]{0.44\textwidth}
         \centering
         \includegraphics[width=.8\linewidth]{mainmatter/images/logouitm.png}
         \caption{$y=x$}
         \label{fig:y_equals_x}
     \end{subfigure} &
     \begin{subfigure}[b]{0.44\textwidth}
         \centering
         \includegraphics[width=.8\linewidth]{mainmatter/images/logouitm.png}
         \caption{$x=y$}
         \label{fig:x_equals_y}
     \end{subfigure} \\
     \hline
    \end{tabular}
    \caption{The Two Figures}
    \label{fig:the2fig}
\end{figure}

\lipsum[1]

\begin{figure}[ht]
    \centering
	\begin{tabular}{|c|c|}
		\hline
		\begin{subfigure}[b]{0.44\textwidth}
			\centering
			\includegraphics[width=.8\linewidth]{mainmatter/images/logouitm.png}
			\caption{$x=y$}
         	\label{fig:x_equals_y4}
		\end{subfigure} & 
				\begin{subfigure}[b]{0.44\textwidth}
			\centering
			\includegraphics[width=.8\linewidth]{mainmatter/images/logouitm.png}
			\caption{$x=y$}
         	\label{fig:xy_equals_y3}
		\end{subfigure}	\\
		\hline
		\begin{subfigure}[b]{0.44\textwidth}
			\centering
			\includegraphics[width=.8\linewidth]{mainmatter/images/logouitm.png}
			\caption{$x=y$}
         	\label{fig:yy1}
		\end{subfigure} & 
				\begin{subfigure}[b]{0.44\textwidth}
			\centering
			\includegraphics[width=.8\linewidth]{mainmatter/images/logouitm.png}
			\caption{$x=y$}
         	\label{fig:xy2}
		\end{subfigure}	\\
		\hline		
	\end{tabular}
    \caption{The Four Figures}
    \label{fig:the4fig}
\end{figure}


\begin{landscape}

\begin{figure}[ht]
    \centering
	\begin{tabular}{|c|c|c|}
		\hline
		\begin{subfigure}[b]{0.44\textwidth}
			\centering
			\includegraphics[width=.8\linewidth]{example-image-a.jpg}
			\caption{$x=y$}
         	\label{fig:ly1}
		\end{subfigure} & 
		\begin{subfigure}[b]{0.44\textwidth}
			\centering
			\includegraphics[width=.8\linewidth]{example-image-b.jpg}
			\caption{$x=y$}
         	\label{fig:ly2}
		\end{subfigure} & 
		\begin{subfigure}[b]{0.44\textwidth}
			\centering
			\includegraphics[width=.8\linewidth]{example-image-c.jpg}
			\caption{$x=y$}
         	\label{fig:ly3}
		\end{subfigure}\\
		\hline
		\begin{subfigure}[b]{0.44\textwidth}
			\centering
			\includegraphics[width=.8\linewidth]{mainmatter/images/logouitm.png}
			\caption{$x=y$}
         	\label{fig:ly4}
		\end{subfigure} & 
		\begin{subfigure}[b]{0.44\textwidth}
			\centering
			\includegraphics[width=.8\linewidth]{mainmatter/images/logouitm.png}
			\caption{$x=y$}
         	\label{fig:ly5}
		\end{subfigure} & 
		\begin{subfigure}[b]{0.44\textwidth}
			\centering
			\includegraphics[width=.8\linewidth]{mainmatter/images/logouitm.png}
			\caption{$x=y$}
         	\label{fig:ly6}
		\end{subfigure}\\
		\hline
		\begin{subfigure}[b]{0.44\textwidth}
			\centering
			\includegraphics[width=.8\linewidth]{example-image-a.jpg}
			\caption{$x=y$}
         	\label{fig:ly7}
		\end{subfigure} & 
		\begin{subfigure}[b]{0.44\textwidth}
			\centering
			\includegraphics[width=.8\linewidth]{example-image-b.jpg}
			\caption{$x=y$}
         	\label{fig:ly8}
		\end{subfigure} & 
		\begin{subfigure}[b]{0.44\textwidth}
			\centering
			\includegraphics[width=.8\linewidth]{example-image-c.jpg}
			\caption{$x=y$}
         	\label{fig:ly9}
		\end{subfigure}\\
		\hline
	\end{tabular}	      
    \caption{The Landscape Figures}
    \label{fig:my_label_landscape}
\end{figure}

\end{landscape}

\lipsum[1]