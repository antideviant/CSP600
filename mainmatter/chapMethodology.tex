\chapter{RESEARCH METHODOLOGY}
\label{ch:method}

\section{Introduction}
This section provides an in-depth description of the design and development characteristics of Alunan, including the tools, processes, and techniques used. The chapter additionally outlines the research technique utilized in the development of the Alunan application and its subsequent implementation inside the project. This project follows the five stages of the Mobile Application Development Lifecycle (MADLC).

\section{Overview of Mobile Application Development Lifecycle (MADLC)}
insert Methodology diagram

\section{Methodology Development and Related Activities}

\subsection{Identification Phase}
insert table

\subsection{Design Phase}
insert table

\subsection{Development Phase}
insert table

\subsection{Prototyping Phase}
insert table

\subsection{Testing Phase}
insert table

\section{Study Area}

\section{Sampling}


\begin{table}[ht]
    \caption{My Sample}
    \begin{tabular}{cc}
        \toprule %header
        \textbf{Millimeters} & \textbf{Centimeters}\\
        \textbf{mm}          &   \textbf{cm}\\
        \midrule
        1           &   0.1\\ \hline
        10          &   1\\ \hline
        100         &   10\\ \hline
        1000        &   100\\ \hline
        10000       &   1000\\
        \bottomrule
    \end{tabular}
    \par\raggedright Note: This table is useful for $\ldots$.
    \label{tab:my_label}
\end{table}

\begin{table}[ht]
    \caption{The Second Sample}
    \begin{tabular}{>{\centering\arraybackslash}p{.47\textwidth} >{\centering\arraybackslash}p{.47\textwidth}}
        \toprule %header
        \textbf{Millimeters} & \textbf{Centimeters}\\
        \textbf{mm}          &   \textbf{cm}\\
        \midrule
        1           &   0.1\\ \hline
        10          &   1\\ \hline
        100         &   10\\ \hline
        1000        &   100\\ \hline
        10000       &   1000\\
        \bottomrule
    \end{tabular}
    \par\raggedright Note: This table is useful for $\ldots$.
    \label{tab:my_second_label}
\end{table}

\begin{figure}[ht]
    \centering
    \fbox{%
        \includegraphics{mainmatter/images/logouitm.png}
    }
    \caption{A New Figure Again!}
    \label{fig:newfig}
\end{figure}

\lipsum[2]

\begin{figure}[ht]
    \centering
    \begin{tabular}{|c|c|}
    \hline
     \begin{subfigure}[b]{0.44\textwidth}
         \centering
         \includegraphics[width=.8\linewidth]{mainmatter/images/logouitm.png}
         \caption{$y=x$}
         \label{fig:y_equals_x}
     \end{subfigure} &
     \begin{subfigure}[b]{0.44\textwidth}
         \centering
         \includegraphics[width=.8\linewidth]{mainmatter/images/logouitm.png}
         \caption{$x=y$}
         \label{fig:x_equals_y}
     \end{subfigure} \\
     \hline
    \end{tabular}
    \caption{The Two Figures}
    \label{fig:the2fig}
\end{figure}

\lipsum[1]

\begin{figure}[ht]
    \centering
	\begin{tabular}{|c|c|}
		\hline
		\begin{subfigure}[b]{0.44\textwidth}
			\centering
			\includegraphics[width=.8\linewidth]{mainmatter/images/logouitm.png}
			\caption{$x=y$}
         	\label{fig:x_equals_y4}
		\end{subfigure} & 
				\begin{subfigure}[b]{0.44\textwidth}
			\centering
			\includegraphics[width=.8\linewidth]{mainmatter/images/logouitm.png}
			\caption{$x=y$}
         	\label{fig:xy_equals_y3}
		\end{subfigure}	\\
		\hline
		\begin{subfigure}[b]{0.44\textwidth}
			\centering
			\includegraphics[width=.8\linewidth]{mainmatter/images/logouitm.png}
			\caption{$x=y$}
         	\label{fig:yy1}
		\end{subfigure} & 
				\begin{subfigure}[b]{0.44\textwidth}
			\centering
			\includegraphics[width=.8\linewidth]{mainmatter/images/logouitm.png}
			\caption{$x=y$}
         	\label{fig:xy2}
		\end{subfigure}	\\
		\hline		
	\end{tabular}
    \caption{The Four Figures}
    \label{fig:the4fig}
\end{figure}


\begin{landscape}

\begin{figure}[ht]
    \centering
	\begin{tabular}{|c|c|c|}
		\hline
		\begin{subfigure}[b]{0.44\textwidth}
			\centering
			\includegraphics[width=.8\linewidth]{example-image-a.jpg}
			\caption{$x=y$}
         	\label{fig:ly1}
		\end{subfigure} & 
		\begin{subfigure}[b]{0.44\textwidth}
			\centering
			\includegraphics[width=.8\linewidth]{example-image-b.jpg}
			\caption{$x=y$}
         	\label{fig:ly2}
		\end{subfigure} & 
		\begin{subfigure}[b]{0.44\textwidth}
			\centering
			\includegraphics[width=.8\linewidth]{example-image-c.jpg}
			\caption{$x=y$}
         	\label{fig:ly3}
		\end{subfigure}\\
		\hline
		\begin{subfigure}[b]{0.44\textwidth}
			\centering
			\includegraphics[width=.8\linewidth]{mainmatter/images/logouitm.png}
			\caption{$x=y$}
         	\label{fig:ly4}
		\end{subfigure} & 
		\begin{subfigure}[b]{0.44\textwidth}
			\centering
			\includegraphics[width=.8\linewidth]{mainmatter/images/logouitm.png}
			\caption{$x=y$}
         	\label{fig:ly5}
		\end{subfigure} & 
		\begin{subfigure}[b]{0.44\textwidth}
			\centering
			\includegraphics[width=.8\linewidth]{mainmatter/images/logouitm.png}
			\caption{$x=y$}
         	\label{fig:ly6}
		\end{subfigure}\\
		\hline
		\begin{subfigure}[b]{0.44\textwidth}
			\centering
			\includegraphics[width=.8\linewidth]{example-image-a.jpg}
			\caption{$x=y$}
         	\label{fig:ly7}
		\end{subfigure} & 
		\begin{subfigure}[b]{0.44\textwidth}
			\centering
			\includegraphics[width=.8\linewidth]{example-image-b.jpg}
			\caption{$x=y$}
         	\label{fig:ly8}
		\end{subfigure} & 
		\begin{subfigure}[b]{0.44\textwidth}
			\centering
			\includegraphics[width=.8\linewidth]{example-image-c.jpg}
			\caption{$x=y$}
         	\label{fig:ly9}
		\end{subfigure}\\
		\hline
	\end{tabular}	      
    \caption{The Landscape Figures}
    \label{fig:my_label_landscape}
\end{figure}

\end{landscape}

\lipsum[1]