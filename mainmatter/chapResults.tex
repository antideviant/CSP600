\chapter{RESULTS AND DISCUSSIONS}
\label{ch:results}

\section{Introduction}
This chapter defines and explores the deliverables and objectives following their particular goals. This chapter extensively examines various aspects of the Alunan mobile application development process, including detailed descriptions of system requirements, user personas, storyboard creation, flowcharts, use case diagrams, database configurations, and the evolution of designs from low-fidelity to high-fidelity.

\section{Identification of System Requirements}
The process of identifying system requirements is comprehensive, requiring the collection, analysis, and documentation of the constraints and demands of a system or application \parencite{mokos20}. Generally, this process starts with stakeholder engagement to gain insight into the business objectives. Subsequently, requirements are obtained through workshops, surveys, and interviews. The examination of these requirements entails the recognition of connections, priorities, and possible conflicts; documentation serves to guarantee clear communication and consistency among the parties involved. Furthermore, the incorporation of validation and verification mechanisms serves to guarantee that the specified requirements precisely mirror the intended functionalities and limitations of the system. \\

The creation of system requirements is critical to the achievement of project objectives, as they provide the fundamental structure that guides the entirety of the development process. They offer designers, developers, and testers a strategic plan that directs their efforts toward developing a solution that satisfies the requirements of stakeholders. Incomplete or ambiguous requirements may necessitate expensive iterations of validation testing and design modifications in later stages. The unresolved difficulty of validating and refining system requirements early on emphasizes the significance of system requirements throughout the development process. \textcite{mokos20} stated that efficient and precisely defined requirements minimize the likelihood of costly rework or project failure, and promote clear communication among project teams. In simple terms, comprehensive system requirements increase the probability of successfully implementing a solution that fulfills the demands of stakeholders, accomplishes organizational goals, and provides value.

\subsection{System Requirements}
System requirements, as elucidated by \textcite{mokos20}, delineate the essential conditions and functionalities that a system must possess to meet the objectives and expectations of stakeholders. They function as a strategic plan that directs the development process by defining the intended functionalities, standards for performance, and limitations. Although frequently expressed in informal terms, system requirements may contain uncertainties, which calls for their formalization to achieve clarity and accuracy. \\

Addressing the development of the Alunan mobile application, this section provides a comprehensive overview of the system's requirements. Based on user expectations, system requirements are the capabilities that the system might be required to have The generation of potential solutions to a problem begins with the creative processes of ideation and brainstorming. By employing judgment and deduction, an extensive inventory of suggestions can be generated, a portion of which might even surpass the original ideas in terms of originality or unconventionality.

\begin{enumerate}[A.]
    \item \textbf{Functional Requirements and Non-Functional Requirements} \\ Functional requirements are the features or functions that developers must implement for users to complete their tasks. Meanwhile, non-functional requirements specify the attributes, characteristics, and constraints that govern the overall behavior, usability, and performance of a system, beyond its functional capabilities. These tasks should correspond to the problem statement presented in Chapter 1. Tables 4.1 and 4.2 show the functional and non-functional requirements for the Alunan mobile application. \\
    
    \begin{table}[htb]
        \caption{List of Functional Requirements for the Alunan Mobile Application}
        \label{tab:mytable}
        \centering
        \begin{tabular}{|p{7.5cm}|p{9.5cm}|}
        \hline
        \multicolumn{1}{|c|}{\textbf{Functional Requirements}} & 
        \multicolumn{1}{c|}{\textbf{Descriptions}} \\
        \hline 
        \multicolumn{1}{|c|}{Authentication} & To use Alunan, users (musicians or enthusiasts) are required to register and log in. \\ \hline
        \multicolumn{1}{|c|}{User Profile} & Users (both Musicians and Enthusiasts) should be able to create and manage their profiles, including updating details such as profile picture, full name, email, and password. \\ \hline
        \multicolumn{1}{|c|}{Music Snippet Sharing} & Musicians should have the capability to create posts with a description and a URL link to share music snippets on their home page. \\ \hline
        \multicolumn{1}{|c|}{Favourite / Bookmark} & Enthusiasts can view posts from all musicians on their 'Post' page and can favorite/bookmark/follow any musician. There should be a separate 'Favourite' page to display posts only from favorited musicians. \\ \hline
        \multicolumn{1}{|c|}{Music Ratings and Reviews} & Enthusiasts should be able to rate musician posts (1 to 5 stars) with a review (up to 150 characters). Additionally, Enthusiasts should have a 'My Reviews' page to view posts they have reviewed, while Musicians should have a 'Reviews' page to display all ratings and reviews received from Enthusiasts. \\ \hline
        \end{tabular}
    \end{table}

    \begin{table}[htb]
        \caption{List of Non-Functional Requirements for the Alunan Mobile Application}
        \label{tab:mytable}
        \centering
        \begin{tabular}{|p{8.5cm}|p{8.5cm}|}
        \hline
        \multicolumn{1}{|c|}{\textbf{Non-Functional Requirements}} & 
        \multicolumn{1}{c|}{\textbf{Descriptions}} \\
        \hline 
        \multicolumn{1}{|c|}{Platform Compability} & To use Alunan, users (musicians or enthusiasts) are required to register and log in. \\ \hline
        \multicolumn{1}{|c|}{Performance and Responsiveness} & Users (both Musicians and Enthusiasts) should be able to create and manage their profiles, including updating details such as profile picture, full name, email, and password. \\ \hline
        \multicolumn{1}{|c|}{Usability} & Musicians should have the capability to create posts with a description and a URL link to share music snippets on their home page. \\ \hline
        \multicolumn{1}{|c|}{Reliability} & Enthusiasts can view posts from all musicians on their 'Post' page and can favorite/bookmark/follow any musician. There should be a separate 'Favourite' page to display posts only from favorited musicians. \\ \hline
        \multicolumn{1}{|c|}{Security} & Enthusiasts should be able to rate musician posts (1 to 5 stars) with a review (up to 150 characters). Additionally, Enthusiasts should have a 'My Reviews' page to view posts they have reviewed, while Musicians should have a 'Reviews' page to display all ratings and reviews received from Enthusiasts. \\ \hline
        \end{tabular}
    \end{table}
    \pagebreak

    \item \textbf{IT Infra Components} \\
    IT infrastructure components for mobile application development include databases, APIs, cloud services, and development tools necessary to create, deploy, and maintain mobile applications. Hardware and software are the two primary teams of components forming the information technology infrastructure, which is composed of mutually beneficial elements. The hardware and software utilized by Alunan: A Mobile Application for Local Musicians' Online Community and Music Discovery are detailed in Tables 4.3 and 4.4. It will specify the components of the IT infrastructure necessary for the mobile application to function. \\

    \begin{table}[htb]
        \caption{List of Hardware for Alunan Mobile Application} 
        \label{tab:hardware}
        \centering
        \begin{tabular}{|c|p{3cm}|p{4cm}|p{5cm}|}
        \hline
        \textbf{No} & \textbf{Purpose} & \textbf{Hardware} & \textbf{Specification} \\
        \hline 
        1 & To run the Alunan mobile application on Android operating system & \vspace{0.1cm} \includegraphics[width=4cm,height=4.5cm]{mainmatter/images/perantisiswa.jpg} & \multirow{5}{4cm}{Model: Samsung Galaxy Tab A8 LTE \\ Operating System: Android 14, One UI 6.0 \\ Processor: ARM Octa-Core A75 2.0GHz \\ Memory: 4GB \\ Storage: 64GB \\ Display: TFT 10.5 inches (1920 x 1200 pixel)} \\
        \cline{1-4}
        \end{tabular}
    \end{table}

    \begin{table}[htb]
        \caption{List of Software for Alunan Mobile Application} 
        \label{tab:software}
        \centering
        \begin{tabular}{|p{0.5cm}|p{3.5cm}|p{9cm}|}
        \hline
        \multicolumn{1}{|c|}{\textbf{No}} & 
        \multicolumn{1}{c|}{\textbf{Software}} & 
        \multicolumn{1}{c|}{\textbf{Descriptions}} \\
        \hline
        \multirow{1}{*}{%
            \begin{tabular}[c]{@{}p{0.5cm}@{}}
            \raggedright 1 \\
            \end{tabular}
        } &
        \multirow{1}{*}{%
            \begin{tabular}[c]{@{}p{2.5cm}@{}}
            \raggedright Figma \\
            \end{tabular}
        } &
        \multirow{1}{*}{%
            \begin{tabular}[c]{@{}p{10cm}@{}}
            \raggedright To create the user interface for the application \\
            \end{tabular}
        } \\ \hline
        \multirow{1}{*}{%
            \begin{tabular}[c]{@{}p{0.5cm}@{}}
            \raggedright 2 \\
            \end{tabular}
        } &
        \multirow{1}{*}{%
            \begin{tabular}[c]{@{}p{2.5cm}@{}}
            \raggedright Canva \\
            \end{tabular}
        } &
        \multirow{1}{*}{%
            \begin{tabular}[c]{@{}p{10cm}@{}}
            \raggedright To create a user persona for this project \\
            \end{tabular}
        } \\ \hline
        \multirow{1}{*}{%
            \begin{tabular}[c]{@{}p{0.5cm}@{}}
            \raggedright 3 \\
            \end{tabular}
        } &
        \multirow{1}{*}{%
            \begin{tabular}[c]{@{}p{2.5cm}@{}}
            \raggedright Lucidchart \\
            \end{tabular}
        } &
        \multirow{1}{*}{%
            \begin{tabular}[c]{@{}p{10cm}@{}}
            \raggedright To create a Gantt chart for the project \\
            \end{tabular}
        } \\ \hline
        \multirow{1}{*}{%
            \begin{tabular}[c]{@{}p{0.5cm}@{}}
            \raggedright 4 \\
            \end{tabular}
        } &
        \multirow{1}{*}{%
            \begin{tabular}[c]{@{}p{2.5cm}@{}}
            \raggedright Draw.io \\
            \end{tabular}
        } &
        \multirow{1}{*}{%
            \begin{tabular}[c]{@{}p{10cm}@{}}
            \raggedright To create the interaction flow for the application  \\
            \end{tabular}
        } \\ \hline
        \multirow{1}{*}{%
            \begin{tabular}[c]{@{}p{0.5cm}@{}}
            \raggedright 5 \\
            \end{tabular}
        } &
        \multirow{1}{*}{%
            \begin{tabular}[c]{@{}p{2.5cm}@{}}
            \raggedright Microsoft Word \\
            \end{tabular}
        } &
        \multirow{1}{*}{%
            \begin{tabular}[c]{@{}p{10cm}@{}}
            \raggedright To create the documentation and reports of the project \\
            \end{tabular}
        } \\ \hline
        \multirow{1}{*}{%
            \begin{tabular}[c]{@{}p{0.5cm}@{}}
            \raggedright 6 \\
            \end{tabular}
        } &
        \multirow{1}{*}{%
            \begin{tabular}[c]{@{}p{4cm}@{}}
            \raggedright Visual Studio Code \\
            \end{tabular}
        } &
        \multirow{1}{*}{%
            \begin{tabular}[c]{@{}p{10cm}@{}}
            \raggedright To create the documentation and reports of the project \\
            \end{tabular}
        } \\ \hline
        \multirow{1}{*}{%
            \begin{tabular}[c]{@{}p{0.5cm}@{}}
            \raggedright 7 \\
            \end{tabular}
        } &
        \multirow{1}{*}{%
            \begin{tabular}[c]{@{}p{2.5cm}@{}}
            \raggedright Android Studio \\
            \end{tabular}
        } &
        \multirow{1}{*}{%
            \begin{tabular}[c]{@{}p{10cm}@{}}
            \raggedright To develop the front-end and back-end of the application \\
            \end{tabular}
        } \\ \hline
        \multirow{1}{*}{%
            \begin{tabular}[c]{@{}p{0.5cm}@{}}
            \raggedright 8 \\
            \end{tabular}
        } &
        \multirow{1}{*}{%
            \begin{tabular}[c]{@{}p{2.5cm}@{}}
            \raggedright Firebase \\
            \end{tabular}
        } &
        \multirow{1}{*}{%
            \begin{tabular}[c]{@{}p{10cm}@{}}
            \raggedright To develop a database of the application \\
            \end{tabular}
        } \\ \hline
        \multirow{1}{*}{%
            \begin{tabular}[c]{@{}p{0.5cm}@{}}
            \raggedright 9 \\
            \end{tabular}
        } &
        \multirow{1}{*}{%
            \begin{tabular}[c]{@{}p{2.5cm}@{}}
            \raggedright GitHub \\
            \end{tabular}
        } &
        \multirow{1}{*}{%
            \begin{tabular}[c]{@{}p{10cm}@{}}
            \raggedright To control version of the application and report \\
            \end{tabular}
        } \\ \hline
        \end{tabular}
    \end{table} 
\end{enumerate}
\pagebreak

\section{Designing Alunan: A Mobile Application for Local Musicians’ Online
Community and Music Discovery}
This section explores the crucial structure plan required to ensure a consistent and efficient mobile application. This section also addresses Objective 2 from Chapter 1, which involves to design Alunan as a mobile application for local independent musicians’ online community and music discovery. This section includes user personas, a SCAMPER technique table, a storyboard, a flowchart, a Use Case Diagram, a Hierarchical Model, a low-fidelity prototype, and a medium-fidelity prototype. The study design and diagrams will offer a thorough overview and explanation of the project's development design.

\subsection{User Persona}
\begin{figure}[h]
    \centering
    \includegraphics[width=0.9\linewidth]{mainmatter/images/userpersona1.png}
	\caption{Figure of User Persona 1 (Enthusiast)}
    \caption*{\textit{Image Source: Unsplash - Beautiful Free Images \& Pictures [https://unsplash.com/]}}
    \label{fig:myfig40}
\end{figure}

\begin{figure}[h]
    \centering
    \includegraphics[width=0.9\linewidth]{mainmatter/images/userpersona2.png}
	\caption{Figure of User Persona 2 (Musician)}
    \caption*{\textit{Image Source: The image used in this persona is from a real-life person and is being consented to be used in this project.}}
    \label{fig:myfig41}
\end{figure}
\pagebreak

\subsection{SCAMPER Technique}
\textit{Purpose:} SCAMPER is a technique for looking at possible transformations that you could apply to a product or process. \parencite{santos15} Looking at these transformations can help you identify "out-of-the-box" approaches by looking at the problem from different perspectives. It is particularly useful where conventional approaches to the problem may already have been tried unsuccessfully. SCAMPER stands for: Substitute, Combine, Adapt, Modify/Magnify/Minify, Put to other uses, Eliminate, Rearrange/Reverse.
\pagebreak

\textit{Process/Product Reviewed:} Bandcamp
\begin{longtable}{|c|p{3cm}|p{4.5cm}|p{4.5cm}|}
\caption{SCAMPER Technique Table}
\label{tab:my_table}\\
\hline
\textbf{} & \textbf{Transformation} & \textbf{Typical questions} & \textbf{Solution Ideas} \\
\hline
\endfirsthead
\multicolumn{4}{c}{{\tablename\ \thetable{} -- Continued from previous page}} \\
\hline
\textbf{} & \textbf{Transformation} & \textbf{Typical questions} & \textbf{Solution Ideas} \\
\hline
\endhead

S & Substitute & What can Who: can I substitute to make an improvement? What happens if I swap X for Y? How can I substitute the place, time, materials or people? & Substitute the traditional email/password authentication with social media login options for quicker and easier access. \\
\hline
C & Combine & What materials, features, processes, people, products or components can I combine within the problem area? Where can I build synergy with other products/processes? & Combine the user profile settings and the music snippet sharing feature into a single interactive dashboard for musicians, making it more convenient for them to manage their profiles and share music snippets. \\
\hline
A & Adapt & What other products/processes are similar to the one at root cause of our problem? What if we adapted them? What could we change to make them fit our purpose? & Adapt the favorite/bookmark feature to incorporate a recommendation algorithm, suggesting musicians to enthusiasts based on their listening preferences. \\
\hline
M & Modify & What ways can we completely change the product/process? Can it be improved by making it stronger, larger, higher, longer, exaggerated or more frequent? Can it be improved by making it smaller, lighter, shorter, less prominent or less frequent? & Modify the review character limit to allow for longer, more detailed reviews, enabling enthusiasts to provide more comprehensive feedback to musicians. \\
\hline
P & Put to other uses & What other products/processes could do what we need to do? What other things are going on that we could make use of it? & Put the music ratings and reviews to other uses by aggregating data to provide insights and analytics for musicians, helping them understand audience preferences and improve their music. \\
\hline
E & Eliminate & What would happen if we remove a component of the product/process? What would happen if we remove the whole thing? How could we achieve the same objective if we weren't able to do this way? & Eliminate redundant features or options in the user interface to streamline the user experience and reduce clutter. \\
\hline
R & Rearrange & What if we reversed the process? What if we did step B before step A? What if we moved step A \& did it last step of step Z first? What if we did these two steps together? & Rearrange the layout of the 'My Reviews' page to prioritize recently reviewed posts, allowing enthusiasts to easily keep track of their recent interactions with musicians. \\
\hline
\multicolumn{4}{l}{\textit{Source: \textcite{santos15}}} \\
\end{longtable}

\subsection{Storyboard Conception}

\subsection{External Device Design Conception}
\begin{enumerate}[A.]
    \item \textbf{IT Infra Architecture Diagram} \\
    Lorem ipsum dolor
\end{enumerate}

\subsection{Flowchart of Alunan: A Mobile Application for Local Musicians’ Online Community and Music Discovery}

\subsection{Use Case Diagram of Alunan: A Mobile Application for Local Musicians’ Online Community and Music Discovery}

\subsection{Data Design for Alunan: A Mobile Application for Local Musicians’ Online Community and Music Discovery}
\begin{enumerate}[A.]
    \item \textbf{Hierarchical Model Diagram} \\
    Lorem ipsum dolor
\end{enumerate}

\section{Developing Alunan: A Mobile Application for Local Musicians’ Online Community and Music Discovery}
\subsection{Front-End Development for Alunan: A Mobile Application for Local Musicians’ Online Community and Music Discovery}

%% An example of a long table + landscape orientation
\begin{landscape}

\begin{longtable}{c|c|c|c|c|c} % Need to have 6 c's if there are 6 columns
\caption{A long table.} \label{tab:alongtablelandscape} \\

% header for 1st page
\hline\multicolumn{1}{c|}{\textbf{First column}} &
 \multicolumn{1}{c|}{\textbf{Second column}} &
 \multicolumn{1}{c|}{\textbf{Third column}} &
 \multicolumn{1}{c|}{\textbf{Forth column}} & 
 \multicolumn{1}{c|}{\textbf{Fifth column}} & 
 \multicolumn{1}{c}{\textbf{Last column}}
 \\ \hline 
\endfirsthead

% What to say at the beginning of the table on the next page, if more than 1 pages.
% Change 6 to the number of columns.
% Comment out these two lines if not needed.
\multicolumn{6}{c}%
{{\bfseries \tablename\ \thetable{} -- continued from previous page}} \\

% header if more than 1 pages.
\hline \multicolumn{1}{c|}{\textbf{First column}} & \multicolumn{1}{c|}{\textbf{Second column}} & \multicolumn{1}{c}{\textbf{Third column}} \\ \hline

%DO NOT REMOVE THIS LINE
\endhead

% What to say at the end of the table, if more than 1 pages.
% Change 6 to the number of columns.
% Comment out this line if not needed.
\hline \multicolumn{6}{|r|}{{Continued on next page}} \\ \hline

%DO NOT REMOVE THIS LINE
\endfoot

% repeat \hline if require more than one line at the end of the table
\hline
\endlastfoot

One & Two & 10.2345667890122 & a & 8901223532523 2334235235 & 2334235235 abcdefghijkl 123456\\ \hline
One & Two & 10.2345667890122 & a & 8901223532523 2334235235 & 2334235235 abcdefghijkl 123456\\ \hline
One & Two & 10.2345667890122 & a & 8901223532523 2334235235 & 2334235235 abcdefghijkl 123456\\ \hline
One & Two & 10.2345667890122 & a & 8901223532523 2334235235 & 2334235235 abcdefghijkl 123456\\ \hline
One & Two & 10.2345667890122 & a & 8901223532523 2334235235 & 2334235235 abcdefghijkl 123456\\ \hline
One & Two & 10.2345667890122 & a & 8901223532523 2334235235 & 2334235235 abcdefghijkl 123456\\ \hline
One & Two & 10.2345667890122 & a & 8901223532523 2334235235 & 2334235235 abcdefghijkl 123456\\ \hline
One & Two & 10.2345667890122 & a & 8901223532523 2334235235 & 2334235235 abcdefghijkl 123456\\ \hline
One & Two & 10.2345667890122 & a & 8901223532523 2334235235 & 2334235235 abcdefghijkl 123456\\ \hline
One & Two & 10.2345667890122 & a & 8901223532523 2334235235 & 2334235235 abcdefghijkl 123456\\ \hline
One & Two & 10.2345667890122 & a & 8901223532523 2334235235 & 2334235235 abcdefghijkl 123456\\ \hline
One & Two & 10.2345667890122 & a & 8901223532523 2334235235 & 2334235235 abcdefghijkl 123456\\ \hline
One & Two & 10.2345667890122 & a & 8901223532523 2334235235 & 2334235235 abcdefghijkl 123456\\ \hline
One & Two & 10.2345667890122 & a & 8901223532523 2334235235 & 2334235235 abcdefghijkl 123456\\ \hline
One & Two & 10.2345667890122 & a & 8901223532523 2334235235 & 2334235235 abcdefghijkl 123456\\ \hline
One & Two & 10.2345667890122 & a & 8901223532523 2334235235 & 2334235235 abcdefghijkl 123456\\ \hline
One & Two & 10.2345667890122 & a & 8901223532523 2334235235 & 2334235235 abcdefghijkl 123456\\ \hline
One & Two & 10.2345667890122 & a & 8901223532523 2334235235 & 2334235235 abcdefghijkl 123456\\ \hline
One & Two & 10.2345667890122 & a & 8901223532523 2334235235 & 2334235235 abcdefghijkl 123456\\ \hline
One & Two & 10.2345667890122 & a & 8901222r24234 2334235235 & 2334232235 abcdefghijkl 123456\\
\end{longtable}

\end{landscape}

\lipsum[1]