\chapter{INTRODUCTION}
\label{ch:intro}
An outline of this project is given in this chapter. This chapter includes the problem background, problem statement, project aim, objective, scope, significance of the study and chapter summary.

\section{Problem Background}
The current digital environment reveals symptoms of local independent musicians having a divided online presence, being underrepresented, and lacking effective methods for discovery and marketing. These symptoms point to a broader perceived issue. This issue is defined by the presence of various websites and applications, a lack of features specifically tailored for musicians and fierce competition for visibility. In this particular field, the lack of a specialized online community platform is a notable challenge for local independent musicians who aim to connect, explore, and gain knowledge within the music industry. \\

The evolution of the music scene in Malaysia corresponds to the symptoms and perceived issues, as emphasized by \textcite{ong19}, by emphasizing the cultural and social dynamics of musical expression. The evident importance of digital platforms is highlighted by the influence of alternative media, blogs, and websites in creating the music community. Nevertheless, the divided online existence and insufficient visibility of local musicians indicate the necessity for a unified platform that tackles the issues within the specified problem area. \\

\textcite{mohd21} stresses the importance of conserving Malaysian music traditions while embracing modern influences, thereby contributing to the solution. An online community platform, as suggested in the issue domain, could provide a space where modern and conventional elements can live together. This platform serves as a medium for musicians to authentically express themselves, addressing the absence of promotional resources and promoting a sense of community among music listeners. \\

The difficulties identified by \textcite{haynes18} about the obstacles faced by young musicians in the overcrowded internet music industry are closely linked to the indications of intense rivalry for exposure. The recognition of the critical importance of social media for independent artists corresponds with the perceived issue of inadequate features tailored specifically for musicians. An exclusive online community platform serves as a strategic remedy for these difficulties, offering a unified place that acknowledges the advantages and constraints of social networking for musicians. \\

In short, the efforts made in the field of Malaysian music, motivated by the views mentioned, seek to address the symptoms and perceived issues by promoting the establishment of a specialized online community platform. This platform serves as the central hub for addressing the difficulties faced by local musicians, providing a comprehensive solution within the specific problem area.

\section{Problem Statement}
\begin{figure}[h]
    \centering
    \includegraphics[width=0.6\linewidth]{mainmatter/images/probstate1.png}
    \caption{Challenges in Music Industry Recognition}
    \caption*{\textit{Post by Utusan Malaysia (2022, September 16) [Media Mulia, 2023]}}
    \label{fig:myfig1}
\end{figure}
The primary topic for the problem statement is closely aligned with the findings of \textcite{silahudin19}, who discussed the significance of the music scene's evolution in Malaysia, as well as the cultural and social dynamics of musical expression among specific audiences and musicians. Similar to the challenges faced by folk music in the music industry awards, the present problem relates to the lack of a specialized digital community platform tailored to Malaysian independent musicians. The issue at hand exhibits several clear indications, such as the fragmented digital presence of local independent musicians, their insufficient exposure in the world of online media, and the obvious lack of efficient collaboration and promotion tools. \\

The spread of local independent musicians across multiple websites and applications is one of the principal symptoms. The lack of consistency in their online presence hinders their capacity to establish a strong identity, thereby creating difficulties for supporters and potential collaborators in locating and interacting with them. Moreover, current platforms frequently lack musician-specific functionalities and struggle with intense competition for attention. As a result, a considerable number of highly talented musicians in the area experience a sense of detachment from the music industry and appreciation, which hinders their ability to build significant relationships with their peers and audience. \\

\begin{figure}[h]
    \centering
    \includegraphics[width=0.8\linewidth]{mainmatter/images/probstate2.png}
    \caption{Social Media's Impact on Local Musicians}
    \caption*{\textit{X post by @zharifikml (2023, December 12) [X Corp., 2023]}}
    \label{fig:myfig2}
\end{figure}
The user @zharifikml's concerns expressed on X align with the findings discussed by \textcite{zanuar22} regarding the effectiveness of social media platforms in marketing for independent artists, emphasizing the need for local independent musicians to acquire the skills required to showcase their creations on the internet. The absence of a dedicated platform for musicians may result in cultural loss, as the multifaceted Malaysian music landscape might remain unexplored and uncelebrated. Furthermore, the challenges in social media marketing mentioned in the article parallel the user's concerns about the need for musicians to acquire internet exposure skills. The depreciation of local musical expertise, as mentioned by the user, can adversely affect the financial stability of musicians, echoing the challenges faced by independent artists in using various social media platforms to engage with different segments of their audience. Additionally, the absence of a suitable platform for growth and visibility, as emphasized by the user, aligns with the evolving role of social media in music promotion discussed in the article by \textcite{jarvekulg21}, highlighting the importance of addressing the challenges independent musicians face in the digital age and the need for a dedicated platform to support their growth and visibility. \\

To summarise, the present issue pertains to the lack of a dedicated digital community platform catering to Malaysian independent musicians. It is important to acknowledge and resolve the symptoms and perceived issues associated with this problem domain to facilitate the empowerment of local independent musicians, promote connections between them and their supporters and colleagues, and actively contribute to the conservation and advancement of Malaysia's vast musical heritage.

\section{Project Aim}
This project aims to develop a mobile application named 'Alunan' that serves as an online community and music discovery platform exclusively for local independent musicians and music enthusiasts to connect and discover music within their local music scene.

\section{Project Objectives}
The project objectives are as follows:
\begin{itemize}
    \item To identify system requirements for Alunan as a mobile application for local independent musicians' online community and music discovery.
    \item To design Alunan as a mobile application for local independent musicians' online community and music discovery.
    \item To develop Alunan as a mobile application for local independent musicians' online community and music discovery.
\end{itemize}
\pagebreak

\section{Project Scope}
Below are the scopes of the project:

\subsection{Platform}
\begin{itemize}
    \item Android Studio\\
    This project will be developed using Android Studio as the platform for the mobile application development.
    \item Firebase\\
    This project will be using Firebase as the platform for the database management system.
    \item GitHub\\
    This project will be using GitHub as the platform for the version control system.
    \item Figma\\
    This project will be using Figma as the platform for the user interface design.
\end{itemize}

\subsection{Social Stakeholders}
\begin{itemize}
    \item \textit{Local Independent Musicians}\\
    Alunan provides local independent musicians with a platform to showcase their talent, develop connections with other artists, and promote their music. Users can create profiles, upload music snippets, and obtain useful feedback from the community. Alunan provides networking opportunities, enabling independent musicians to engage with event organizers and industry professionals. Moreover, the platform may provide significant assistance and significant knowledge that encourage the growth and triumph of artists in the music industry, making it a vital asset for both growing and experienced musicians.
    \item \textit{Music Enthusiasts}\\
    Alunan provides an engaging setting for music enthusiasts to delve into their local music scene. Users can discover skilled musicians and music genres, and obtain educational material to enhance their understanding of music. Enthusiasts play a crucial role in boosting local musicians by actively engaging with their favorite local musicians and providing reviews or ratings. Alunan motivates individuals to actively engage in the music industry, establishing a strong sense of inclusion and community among music enthusiasts. Alunan is a platform that enhances the music experience for lovers and supports and celebrates local talent.
\end{itemize}

\subsection{Context of Study}
Alunan focuses on creating a feature-rich mobile platform that meets the particular needs of local independent musicians and music enthusiasts. Features like profile creation are part of the initiative, which helps musicians interact with their fans and create a unique online brand. In order to build anticipation for upcoming releases, musicians can simultaneously share teasers of their songs. By adding favorites and bookmarking any local musicians, users may interact and improve their experience finding new music. Furthermore, by encouraging honest feedback and collaboration, music ratings and reviews help to build a strong local music scene. The goal of the project is to close the gap that exists between performers and fans by encouraging creativity, collaboration, and a greater understanding of local music. \\

To put it briefly, Alunan intends to completely reshape the local music industry by offering a flexible mobile platform that connects artists and fans via profiles, favorites, music snippets, and reviews. This dynamic ecosystem encourages innovation and community involvement.

\subsection{Language}
The Alunan online community mobile application project will primarily use the English language for its interface and communication. This choice of language aims to ensure accessibility and usability for a broad user base, including local musicians and music enthusiasts who are comfortable with English as a means of interaction within the platform. \pagebreak

\subsection{Tools/Equipment Needed for Project}
\begin{enumerate}[A.]
    \item \textit{Hardware}
    \begin{itemize}
        \item Mobile Device: A smartphone or tablet that runs on Android operating system and can support the Alunan mobile application.
    \end{itemize}
    \item \textit{Software}
    \begin{itemize}
        \item Android Studio: A popular integrated development environment (IDE) for Android app development.
        \item Figma: A vector graphics editor and prototyping tool which is primarily web-based.
        \item Firebase: A mobile and web application development platform developed by Firebase, Inc. in 2011, then acquired by Google in 2014.
        \item GitHub: A provider of Internet hosting for software development and version control using Git.
    \end{itemize}
\end{enumerate}

\subsection{Features and Functions}
These are the features and functions of the Alunan mobile application:
\begin{enumerate}[A.]
    \item \textit{User Profile Creation}\\
    The user profile creation process in Alunan, a mobile application designed for the online community and music discovery of local musicians, provides a smooth registration procedure where users provide crucial information such as their email, username, and password. Users have the option to select one of two separate profiles, either as musicians or enthusiasts. This allows for a personalized experience tailored to their preferences. Upon joining, members are granted the freedom to enhance their profiles by incorporating profile images and hyperlinks to their social networking and music streaming platforms. This enables a seamless integration of their Alunan presence with their wider online musical persona. Enthusiasts are motivated by gamification components, which reward them with badges for their contributions to music ratings and reviews. These extensive features collectively improve user involvement, promoting creative expression, connections, and active discovery of local music within the Alunan community.
    \item \textit{Music Snippet Sharing}\\
    Alunan, the mobile application, includes an important function called Music Snippet Sharing. This feature aims to strengthen the online community and make it easier for music enthusiasts to find new music. This feature offers musicians a versatile platform for showcasing their talent by sharing captivating snippets of their musical works. Artists can share snippets of their songs, instrumentals, or melodies, providing an appealing glimpse into their creative realm. These snippets function as auditory previews, enticing users to go deeper into their body of work and cultivating a devoted fan following. In addition, Alunan utilizes the capabilities of Apple Music, Spotify, and SoundCloud APIs, allowing musicians to effortlessly distribute these brief sections on these widely-used music platforms, expanding their audience and visibility both within and outside the Alunan community. Alunan's dedication to supporting and promoting local musicians in their musical efforts is emphasized by this integration.
    \item \textit{Favourite / Bookmark Musician}\\
    The "Favourite / Bookmark Musician" function in Alunan, provides music enthusiasts with an effective way to curate their musical experience. Using this functionality, devoted supporters may choose their preferred musicians within the Alunan community, establishing a customized list of artists they hold in high regard. By adding these performers to their bookmarks, enthusiasts may effortlessly remain informed about their latest artistic works, cultivating a more deeper connection with their musical influences. This function surpasses passive consumption, enabling enthusiasts to actively engage with and provide support to their selected artists. Alunan's dedication to improving the music discovery process and fostering a vibrant online community allows fans to actively support and celebrate their preferred local musicians.
    \item \textit{Music Ratings and Reviews}\\
    The "Music Ratings and Reviews" function of Alunan allows music enthusiasts to engage directly with musicians' discography. Enthusiasts can contribute important suggestions and feedback by rating and reviewing the musical offerings of local musicians. This feature functions as a bridge between artists and their audience, enabling enthusiasts to express their admiration, evaluation, and endorsement of musicians' creations. Musicians, in return, gain advantages from constructive criticism that facilitates their artistic development. Alunan fosters an environment that promotes open communication and cooperation, allowing musicians and lovers to work together in defining the local music scene. Alunan is a dynamic platform that enhances the sense of community and shared passion for music, promoting both discovery and artistic development.
\end{enumerate}

\subsection{Project Limitations}
The limitations of this project are as follows:
\begin{enumerate}[1.]
    \item The Alunan mobile application is exclusively compatible with the Android mobile platform. iOS users would be impacted by this, as they are unable to install the application and must instead use a smartphone running Android. 
    \item Alunan's focus is on serving only local musicians and enthusiasts in Malaysia. Thus, it can limit its growth potential and reach. Expanding beyond this geographic boundary could be challenging.
    \item Restricting the platform to local musicians may limit the diversity of music genres and styles available to users. This could hinder the platform's ability to cater to a wide range of musical tastes.
    \item A geographically limited user base may result in a smaller number of users compared to global music discovery platforms. This could affect the level of engagement and interaction within the community.
\end{enumerate}
\pagebreak

\section{Significance of Study}
The significance of this project is as follows:
\begin{enumerate}[1.]
    \item \textit{Local Independent Musicians} \\
    Alunan provides a dedicated platform for local independent musicians to exhibit their talents, gain exposure, and engage with a community of music enthusiasts who value their artistic works. Such exposure can play a crucial role in supporting emerging musicians in establishing their careers and expanding their reach to a wider audience. Alunan encourages the sharing of music snippets among musicians, enabling them to get feedback. This platform creates an environment that promotes artistic growth and improvement, thereby contributing to the overall development of the local music scene.
    \item \textit{Music Enthusiasts} \\
    Alunan serves as a central platform for music enthusiasts to discover and interact with local musicians and their music, providing them with several advantages. The website provides a customized music exploration experience, enabling enthusiasts to discover local musicians and genres that they may not have otherwise come across. Enthusiasts can actively support and participate with their preferred artists by bookmarking them as favorites and submitting ratings and reviews. This enhances their connection to the local music community.
    \item \textit{Local Music Industry} \\
    The Alunan platform has the potential to greatly enhance the local music business by actively encouraging and supporting local musicians. It offers a virtual platform where musicians can acquire acknowledgment and potentially draw opportunities such as collaborations, live shows, and partnerships with nearby establishments. Alunan's expansion has the potential to support the economic development of the local music industry by generating greater demand for local music, hence leading to increased earnings for musicians, music producers, event organizers, and affiliated enterprises. \pagebreak
    \item \textit{Mobile Application Developers} \\
    The development of the Alunan mobile application presents possibilities for mobile app developers. As the platform progresses and grows, there will continue to be a requirement for technical expertise to improve user experience, integrate new functionalities, and guarantee platform security. The successful outcome of Alunan can also function as a good subject of analysis for mobile application developers, offering unique perspectives on developing platforms that cater to certain communities and promoting user involvement through innovative and interactive elements.
\end{enumerate}

\section{Chapter Summary}
The summary for every chapter in this thesis is as follows:
\begin{enumerate}[1.]
    \item \textit{Chapter 1: Introduction} \\
    This chapter of the introduction provides context for the Alunan project, offering an overview of the problem that the platform seeks to address. It also outlines the objectives, scope, and the project's significance within the realm of the local music community and discovery.
    \item \textit{Chapter 2: Literature Review} \\
    This chapter explores key elements that are crucial for the development of Alunan, a specialized mobile application designed for the local music community. The text dives into a study of the regional music scene, investigating patterns, challenges, and effective marketing strategies. The discussion revolves around the complex characteristics of online communities, user-generated content, the integration of social media, and the incorporation of gamification features. The review also examines the exploration of music, including a wide range of genres, concerns related to underrepresentation, and the influence of popular streaming sites. A comprehensive evaluation of established mobile applications such as Letterboxd, Spotify, Twitch, Bandcamp, and SoundCloud offers valuable guidance and insights for the creation of Alunan. This chapter serves as a crucial resource, establishing the basis for the following chapters and the general development of the Alunan platform.
    \item \textit{Chapter 3: Research Methodology} \\
    This chapter provides an in-depth look at the research and development process carried out in this project. It begins with an introductory section, further expanding into the Mobile Application Development Lifecycle (MADLC), encompassing stages such as identification, design, development, prototyping, and testing. The chapter explores the process of developing the methodology and the activities associated with it. It provides insight into the activities, tools, techniques, hardware, and software used in the project. Furthermore, it provides a summary of the documentation process, guaranteeing transparency and quality.
\end{enumerate}