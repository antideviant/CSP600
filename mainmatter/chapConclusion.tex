\chapter{CONCLUSION AND RECOMMENDATIONS}
\label{ch:conclusion}
This chapter serves as the last section of the thesis report, summarising the process of problem-solving and drawing conclusions from the full project lifecycle of developing the Alunan mobile application. It provides an in-depth summary of the project's findings, beginning with the initial specification of the problem and concluding with the final stage of development. In addition, this chapter outlines the limitations of the project and provides suggestions for future improvements to strengthen and enhance the application.

\section{Conclusion}
\begin{enumerate}[A.]
    \item \textbf{Project Summary} \\
    Alunan is an innovative mobile application specifically developed to meet the requirements of local independent musicians and music enthusiasts. This comprehensive platform enables musicians to create profiles, engage with their followers, and create a unique online brand. Musicians can share a post that links directly to their songs on streaming platforms to generate anticipation for their upcoming releases. Meanwhile, music enthusiasts may select their favorite musicians and save them for future reference to increase their music exploration. Alunan promotes an exciting local music community by combining music ratings and reviews, encouraging open discussions and interaction. The application's primary objective is to establish a connection between musicians and fans, promoting innovation and encouraging active participation from the community. Alunan is a PHP-based application developed using Visual Studio Code and phpMyAdmin. It utilizes InfinityFree for online hosting and GitHub for version control. Alunan aims to transform the local music industry by creating a dynamic ecosystem that assists in connections between artists and fans while empowering local musicians and enhancing the cultural environment by offering a platform that celebrates and promotes local talent.

    \item \textbf{Summary of Project Objectives}
    \begin{enumerate}[1.]
        \item \textit{To identify system requirements for Alunan as a mobile application for local independent musicians’ online community and music discovery.} \\
        The identification of the requirements for the Alunan: A Mobile Application for Local Musicians’ Online Community and Music Discovery, is the result of the first objective. This objective was achieved through a series of actions. The initial phase of the activity involved identifying user requirements and conducting a literature review to gather crucial information for this study and project from various sources, such as scientific publications, articles, and conference proceedings papers from a digital library repository. To fully comprehend the functional requirements that could be incorporated into the Alunan mobile application, it is essential to acquire an in-depth knowledge of the various categories of musicians, current trends in the music scene, and the challenges faced by the music community. These also encompass acquiring knowledge about online community behavior, music exploration approaches, and mobile application development. By utilizing tools like Canva, it is essential to construct two user personas for both musicians and music enthusiasts based on the insights obtained. To ensure adherence to the project timeline, a Gantt Chart (see Appendix A) was created using tools like Lucidchart and Canva for primary planning and implementation. This indicates the successful accomplishment of the first primary objective for this project.

        \item \textit{To design Alunan as a mobile application for local independent musicians’ online community and music discovery.} \\
        The second objective was to design Alunan: A Mobile Application for Local Musicians’ Online Community and Music Discovery. The conclusion was reached through a sequence of activities. The initial process to evaluate the uniqueness of the Alunan mobile application was employing the SCAMPER method, which stands for Substitute, Combine, Adapt, Modify/Magnify, Purpose, Eliminate/Minimize, and Rearrange/Reverse. Before starting the development and testing of the high-fidelity prototype, a low-fidelity prototype was generated using tools like Figma to provide general visualization and understanding of the suitable user interface. To visually represent the user's interaction with the features of the Alunan mobile application, involve a flowchart, use case diagram, and sequence diagrams. These two activities were created by utilizing Lucidchart. Furthermore, the objective of the hierarchical diagram and entity relationship diagram for the database structures of this project was to have a comprehensive understanding of the system's data structure.

        \item \textit{To develop Alunan as a mobile application for local independent musicians’ online community and music discovery.} \\
        The third objective, which was to develop the mobile application for Alunan: A Mobile Application for Local Musicians’ Online Community and Music Discovery, was achieved by using PHP and Bootstrap to design the user interface and functionality of the mobile application. The Alunan mobile application was developed using Visual Studio Code as the development platform. The application's database was implemented using phpMyAdmin on InfinityFree hosting. Alunan uses InfinityFree for web hosting of the database and source code files, while GitHub is used for version control. The final objective was to conduct a set of evaluations for the Alunan mobile application, engaging a minimum of eight to twelve users from the local community of independent musicians and music enthusiasts. These users would offer feedback utilizing SUS techniques. User feedback is collected after the testing process using Google Forms and Google Spreadsheets. The development of Alunan: A Mobile Application for Local Musicians’ Online Community and Music Discovery, has been successfully finished.

    \end{enumerate}
\end{enumerate}

\clearpage

\section{Project Limitations}
The development of the Alunan mobile application is taking place, however it is restricted by certain limitations. However, these limitations can strengthen the potential for advancement and growth. The project has experienced the following challenges and limitations:
\begin{enumerate}[1.]
    \item The application primarily utilizes the English language, which may result in the exclusion of specific user demographics or individuals who lack proficiency in English, hence restricting its accessibility and potential user base.
    \item The Alunan mobile application is only designed for the Android platform, meaning that iOS users cannot use it and must rely on Android smartphones.
    \item The effectiveness of the feedback system in rating and reviewing musicians' posts strongly depends on user engagement. The application's success is contingent on the active participation of local independent musicians and music enthusiasts.
    \item The application is hosted on InfinityFree utilizing a free subscription, hence it is dependent on an internet connection to fetch and submit data from an online database. Occasionally, there may be a delay in the loading process.
    \item The dependence of users on other websites requires redirecting to streaming services like Spotify to listen to shared songs, which adversely impacts user experience and introduces extra steps for users.
    \item The application is developed using PHP and Bootstrap, ensuring compatibility with screens of any size. However, it may only sometimes be optimally compatible with specific screen sizes.
    \item Dependence on external libraries and frameworks can potentially lead to compatibility problems or restrict the ability to customize.
    \item The lack of offline capabilities restricts user involvement in regions with limited or no internet access.
\end{enumerate}

\clearpage

\section{Recommendations for Future Work}
This section aims to outline the developer's alternatives for future updates and improvements to the mobile application. After the development of the Alunan mobile application, the evaluators from previous evaluations offer some suggestions and comments to help improve the project.
\begin{enumerate}[1.]
    \item Implement a feature that enables users to effortlessly switch between Malay and English languages by tapping a single button. This would enhance the accessibility of the application for individuals with multiple languages.
    \item Improve compatibility by ensuring that the application is accessible on several operating systems, such as iOS and others.
    \item Integrate gamification elements, such as a reward or points system, to boost participation and engagement among users.
    \item Upgrade to a premium InfinityFree plan or consider alternative hosting solutions to achieve faster loading times for fetching and retrieving data from the online database.
    \item Implement a feature that allows musicians to post previews of their songs directly within the app, eliminating the need for users to be redirected to external streaming services.
    \item Enhance the user interface and user experience (UI/UX) by considering the use of Flutter for development, ensuring seamless navigation and a more cohesive user experience.
    \item Introduce an offline feature, enabling users to browse posts even without an internet connection.
    \item Add a dark mode option for better viewing during nighttime and low-light conditions.
    \item Implement accessibility features for users with disabilities, such as the ability to increase font size and support for screen readers.
\end{enumerate}